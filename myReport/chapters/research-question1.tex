% A short section which summarises the hypothesis or research question(s) that is addressed in the report. This should be a logical development of the preceding section, and may include a list of concrete objectives for the project.
% Link back to the original proposal posed by Denis.
% Lead in from literature review and link to methodology (next section).
\chapter{Research Questions}

\section {Questions}
\subsection{Project Intent}
Q.1] How can a hybrid NLP approach, consisting of Supervised and Unsupervised models, enable early stage extraction of functional status information from free-text clinical records in the absence of gold-standard annotated datasets?

\subsection{Data Collection and Annotation}
Q.2.1] To what extent can a small manually annotated subset of sentences extracted from clinical free-text notes support the evaluation and expansion of auto-labelling strategies across, large, unlabelled clinical corpora?\\
Q.2.2] What limitations and challenges arise from the sparsity and ambiguity of FSI data in world clinical notes, and how do they affect the reliability of downstream modelling?

\subsection{Model Development and Comparison}
Q.3]  How do different modelling paradigms—such as clustering-based, rule-based, and neural network-based approaches—perform individually and in combination for FSI labelling, in terms of scalability, interpretability, and accuracy?

\subsection{Comparative and Analytical Questions}
Q.4.1] How do supervised and unsupervised models differ in identifying implicit FSI sentences or data that lack keywords or phrases? 

\subsection{Interpretability and Results}
Q.5.1] What insights can be drawn from label distributions, visualizations and model results about the latent structure of FSI-related content in clinical corpora?\\
Q.5.2] how does supervised and unsupervised model performance varies across four ICF categories, and what factors ( e.g., vocabulary range, semantic overlap) influence classification process?

